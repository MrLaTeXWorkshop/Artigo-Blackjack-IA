\section{Trabalhos Correlatos}

    %Nesta seção, artigos relacionados à simulações de memória são apresentados.
    %Por exemplo, para cada artigo, um parágrafo. Ao final, um parágrafo que menciona
    %a diferença entre o artigo em questão e os artigos correlatos.

    Um padrão muito comum encontrado na literatura do assunto que está sendo abordado, 
    é o fato de que os algoritmos de aprendizado de máquina, que usam de técnicas avançadas 
    de inteligência artificial, ainda estão em sua juventude neste campo. Como exemplo, tem o 
    artigo de Raphaela(2015), que utiliza de tais técnicas para formar regras para maximizar 
    as vantagens do jogador em relação ao cassino. Entretanto, os algoritmos em geral chegaram
    a um número pequeno de regras, o que fazia com que certas jogadas não contemplassem as configurações 
    do jogo.Marvin e Fernand(2012) também tentou utilizar uma abordagem moderna utilizando uma 
    arquitetura cognitiva chamada CHREST, entretanto, apesar que tal modelo tenha chegado a bons 
    resultados, foi concluído que tal arquitetura seria melhor utilizada em jogos com um número 
    maior de padrões a serem memorizados e reconhecidos, como o jogo Poker. Por fim temos Robert(2012)
    que além de usar uma abordagem moderna com aprendizado genético, também utilizou de estratégias usando 
    inteligências artificiais baseadas em regras, chegando ao resultado que ainda existe espaço a 
    ser otimizado nas estratégias atuais, enquanto que as abordagens mais tradicionais
    continuam produzindo melhores resultados de maneira geral.

    Todos os trabalhos até então mencionados tentam de alguma forma ou outra 
    criar um jogador inteligente de Blackjack para aplicações, Raphaela(2015) criou 
    uma aplicação em C, enquanto que tanto Marvin e Fernand(2012) e Robert(2012) criam 
    aplicações em Java. Nosso objetivo é também criar um jogo, porém que possa ser feito 
    em uma linguagem de alto nível e que tenha uma utilização mais fácil, por isso a linguagem 
    escolhida foi a GDScript da Godot Engine. Além disso, tentaremos utilizar abordagens baseadas 
    em regras ( Expectiminimax ) como também iremos tentar fazer a aplicação usando Aprendizado por máquina.