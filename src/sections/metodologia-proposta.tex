\section{Metodologia ou Proposta de Arquiteturas}

    Antes de compreender a metodologia aplicada a este estudo, é preciso entender
    a natureza do problema que estudo quer resolver, em outras palavras, é preciso 
    dar uma introdução ao jogo 21 e quais regras do jogo serão usados na simulação 
    feita neste estudo.

\subsection{Entendendo 21 e as regras aplicadas}

    O 21 (também conhecido como Blackjack) é um jogo de cartas francês onde pode ser 
    jogado por um ou mais jogadores sempre todos contra o dealer, e para ganhar 
    é necessário ter uma pontuação maior que a do dealer e menor ou igual a 21.
    Essa pontuação é proveniente dos valores das cartas, onde cartas com figuras exceto os “ás” 
    possuem valor igual a 10, o “ás” pode valer 1 ou 11 caso o valor da mão não
    seja superior a 21, caso contrário o valor do ás se altera para 1 , as demais 
    cartas tem o valor igual ao que está escrito na carta. 
    No início de toda rodada cada jogador faz a sua aposta, e recebe duas 
    cartas viradas para cima, junto a isso o dealer também recebe duas cartas,
    porém somente uma fica virada para o cima e ao final do jogo é revelada. 
    Os jogadores podem executar algumas ações sendo elas:
\begin{itemize}
    \item \textbf{hit:} O jogador solicita mais uma carta ao dealer  
    \item \textbf{Stand:} O jogador não vai mais solicitar cartas 
    \item \textbf{Dobrar:} Somente após receber as duas primeiras cartas o 
    jogador pode dobrar a aposta e em consequência receber uma nova carta.
    \item \textbf{Split:} Após receber a primeira duas cartas e caso elas 
    possuam o mesmo valor, o jogador pode dividir essa mão em duas novas mão
    \item \textbf{Surrender:}  O jogador desiste do jogo e perde metade da aposta  
\end{itemize}

    Para o propósito do nosso trabalho colocamos apenas as duas funções 
    mais simples do jogo, hit e stand. 

    Com o jogo e suas regras introduzidos, também seria importante apresentar a 
    tecnologia utilizado neste trabalho, a Godot Engine e sua linguagem GDScript.

\subsection{Godot Engine}

    A Godot engine\cite{GodotEngine} é um motor gráfico para jogos \textit{cross-platform} de 
    código aberto, tendo sua primeira versão lançada em 2014 pelos Argentinos 
    Juan Linietsky e Ariel Manzur. O propósito da Godot é criar um ambiente 
    completo para o desenvolvimento de jogos 2D e 3D, com licença aberta, logo, 
    os desenvolvedores tem todos os direitos em cima da sua criação (sem \textit{royalties}). 
    Alem disso, ela está disponível para todas as plataformas (Linux, macOs, Windows).

    Uma das grandes vantagens de trabalhar com a Godot Engine, e o motivo pelo qual esta
    foi escolhida para realizar este estudo, se deve ao fato que a Godot Engine facilita imensamente 
    o desenvolvimento dos jogos, com sua interface fácil de aprender e sua linguagem de programação
    GDScript, que possui um sistema de tipagem bem parecido com linguagens de alto nível 
    como Python e Javascript. Por fim, a Godot Engine ainda possui suporte a GDNative, permitindo 
    que os jogos sejam programados em C ou C++ e pacotes da comunidade como o Godot-python\cite{GodotPython}, permite 
    rodar códigos de Python dentro da engine, abrindo as portas para o uso de aprendizado de máquina e 
    aprendizado profundo usando pacotes como o Godot RL Agents\cite{beeching2021godotrlagents}.


    %A metodologia utilizada no artigo consiste no estudo de artigos que se correlacionam com o tema estudado, 
    %como também a utilização e desenvolvimento de arquiteturas feitas no simulador Amnesia.
    