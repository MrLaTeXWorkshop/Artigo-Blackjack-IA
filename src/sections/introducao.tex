\section{Introdução}

    %Espera-se que nesta seção, seja escrita uma contextualização inicial 
    %para situar o leitor. Ou seja, saber em qual área ele está "pisando".
    %No contexto discutido, é apresentado um problema em aberto (um problema 
    %pode ser escrito como uma pergunta), em seguida um objetivo para "responder" 
    %ao problema descrito. Depois, a contribuição científica é descrita, e por fim, 
    %um parágrafo para descrever a organização do artigo é feito.

    \cite{Sibgrapi2021}.

    No ramo de inteligência artificial, existe um mercado amplo onde 
    têcnicas dessa área são usadas para criarem oponentes que possam
    derrotar qualquer jogador. Um dos maiores desafios desta divisão,
    é como tratar os jogos de azar, onde a sorte é um elemento que 
    precisa ser calculado para a realização das jogadas.

    Neste estudo, pretendemos criar uma aplicação prática que 
    possa ser usada para criar um jogador inteligente de Blackjack(no 
    Brasil conhecido como 21). Além disso iremos avaliar diferentes 
    estratégias/políticas e avaliar a utilidade esperada de cada estratégia 
    para a inteligência artificial.

    O artigo está dividido da seguinte maneira: primeiros na secção 2 
    será explorado um pouco da literatura já existente do assunto. Na secção 
    3 as regras do Blackjack serão expostas e analisadas no contexto da aplicação 
    que será feita. No estudo em questão, a Godot Engine será utilizada para 
    criar a interface do programa. Na secção 4 será analisado as diferentes 
    estratégias e políticas utilizadas no programa. Na secção 5 acontecerá uma análise 
    dos resultados chegados, a precisão de cada modelo e o tempo de execução. Por fim, 
    terá a conclusão, onde será dado uma opinião final sobre os dados chegados, além de possíveis 
    futuros trabalhos que podem ser feitos a partir deste.