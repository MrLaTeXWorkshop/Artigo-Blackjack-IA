\section{Conclusões}

    %Nas conclusões, inicia-se com um breve resumo do que foi o trabalho desenvolvido,
    %depois faz-se uma discussão acerca dos objetivos alcançados. Ou seja, os objetivos
    %apresentados na Introdução casam com os resultados discutidos na seção de Resultados? 
    %Quais resultados podem ser destacados? Por fim, há um parágrafo para apresentar possíveis trabalhos futuros.

    Após a realização dos testes em todos os cenários propostos, e uma análise dos gráficos apresentados, houve um 
    receio de que o ponto da pesquisa não havia sido alcançado, já que os resultados se mostraram pouco expressivos. 
    Em nosso caso, tanto o LFU quanto o FIFO tiveram um desempenho com diferença menor que 5\% entre eles, em alguns 
    casos nem havendo diferença alguma. Porém, após uma análise e estudo dos artigos correlacionados, foi possível observar que 
    a política de substituição LRU se mostra de forma pouco significativa mais eficiente do que a política FIFO. 

    Tudo que descobrimos esteve desde Smith e Goodman \cite{b1} que disseram que não há diferença de desempenho entre os algoritmos 
    e que aleatoriedade é mais eficiente. Enquanto no artigo da Samira Mirbagher Ajorpaz, Elba Garza, Sangam Jindal, e Daniel A. Jiménez\cite{b3}, 
    mostrou que o desempenho do LRU é pouco superior ao FIFO, algo que consegue ser detectado, mas seu uso diário não teria melhora perceptível e 
    que seria mais importante considerar outros tantos algoritmos que possuem um desempenho significativamente superior aos dois anteriores. Por fim, 
    o artigo de Aurore Junier, Damien Hardy, Isabelle Puaut \cite{b2}, mostrou algo inusitado, indo na direção oposta dos outros dois artigos, 
    apresentou dados de que o LRU pode sim ter um desempenho superior à inserção aleatória de dados na cache.

    Concluindo, nossos testes demonstraram que não há grande diferença desses algoritmos quando usados em caches pequenas, os dois primeiros artigos
    concordam com nosso posicionamento, enquanto o outro artigo discordou. Dessa forma, é importante manter essa discussão aberta para novos experimentos.
    Considerando que cenários de testes diferentes apresentam resultados completamente dispersos.
